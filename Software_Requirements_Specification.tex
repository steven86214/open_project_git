\documentclass{scrreprt}

\usepackage{listings}
\usepackage{underscore}
\usepackage[bookmarks=true]{hyperref}
\usepackage[utf8]{inputenc}
\usepackage[english]{babel}
\usepackage{CJKutf8}
\usepackage{latexsym}

\def\myversion{1.0 }
\date{}

\usepackage{hyperref}
\begin{document}

\begin{flushright}
    \rule{16cm}{5pt}\vskip1cm
    \begin{bfseries}
        \Huge{SOFTWARE REQUIREMENTS\\ SPECIFICATION}\\
        \vspace{1.9cm}
        for\\
        \vspace{1.9cm}
        $<$Final Project$>$\\
        \vspace{1.9cm}
        Prepared by $<$Team 2$>$\\
        \vspace{1.9cm}
        \today\\
    \end{bfseries}
\end{flushright}

\tableofcontents


\chapter{Introduction}

\section{Purpose}
%此model的用途
\begin{CJK}{UTF8}{bkai}
	該模型主要用於識別不同種類的花卉。\\
	有了這個模型,我們就不必記住花卉大量且不同的特徵,能夠輕易地識別出不同的品種同時也具有較高的辨別力。
\end{CJK}

\section{Intended Audience and Reading Suggestions}
%目標使用者及閱讀建議
\begin{CJK}{UTF8}{bkai}
	主要面向為對於花卉鑑賞感興趣, 但無一定基礎的人,也可以用於教學。\\
如果數據量足夠大,或許能夠突變種是從哪種植物轉變而來,對植物相關領域有所幫助。
\end{CJK}

\section{Project Scope}
\begin{CJK}{UTF8}{bkai}
為了講求辨識速度及控制model大小,以元智校園內出現的花朵為主,來訓練model。且訓練的data也是直接在學校內取材,以免因環境差異變化過大,導致辨識率下降。
\end{CJK}

\chapter{Overall Description}

\section{Product Perspective}
%產品視角
\begin{CJK}{UTF8}{bkai}
只要有行動裝置就能夠隨時辨識想要了解的花朵種類。
\end{CJK}

\section{Product Functions}
%產品功能.
\begin{CJK}{UTF8}{bkai}
	從App開啟手機相機對準要辨識的花朵,系統偵測到後,
\end{CJK}

\section{User Classes and Characteristics}
%用戶群跟特徵

\section{Operating Environment}
%操作環境
Smart Phone with Android OS ver.8.0 

\section{Design and Implementation Constraints}
%設計和實作問題
\begin{CJK}{UTF8}{bkai}
	因本專題欲專注於辨識校園內的花朵,所以在資料收集方面只有拍攝校園內的花朵之外貌,因此,本模型在對同種花但是長相有很大的差異的花(花在不同環境下,顏色、樣貌等會略有不同),預測準度及信心度會較低。 \\
	但是,從另一個角度來說,沒有其他環境下生長的花的資料,反而更能提升此模型對本校園中生長的花的預測準度。
\end{CJK}

\section{Assumptions and Dependencies}
%假設及依據
\begin{CJK}{UTF8}{bkai}
	本專題原本採用InceptionV3 的影像辨識模型,但是若要將他移植到APP上,model size太大了,且InceptionV3架構過於龐大,可能不利於行動裝置的即時辨識,因此我們改用MobileNet模型,其缺點就是準確度會稍比InceptionV3低,但是換來了更快的速度跟更小的空間。
\end{CJK}

\chapter{External Interface Requirements}

\section{User Interfaces}
%使用者介面

\section{Hardware Interfaces}
%硬體介面
ASUS_Z012DA (Smart Phone)

\section{Software Interfaces}
%軟體介面
\subsection {tensorflow package}  


\subsection {android studio}



\chapter{System Features}

\section{Description and Priority}
%描述及優先權
%example: 用戶可以通過向網格添加課程對象來創建季度課程計劃,其中的列表示季度。 可以刪除課程並通過拖放在四分之一之間移動。 可以編輯課程數據。 用戶可以指定必備和必修課程以及課程實現的單元數。
Users may create a quarterly course schedule by adding course objects to a grid, the columns of which represent quarters. Courses may be deleted and moved between quarters via drag-and-drop. Course data may be edited. Users may specify prerequisite and corequisite courses as well as the number of units the course fulfills.

\section{Stimulus/Response Sequences}


\section{Functional Requirements}
%功能需求
\begin{CJK}{UTF8}{bkai}
	判斷出元智大學校園的花的種類。
\end{CJK}

\chapter{Other Nonfunctional Requirements}

\section{Performance Requirements}
%性能需求
\begin{CJK}{UTF8}{bkai}
	即時的(小於1s),準確的判斷出元智大學校園內花的種類( 趨近100\% )。
\end{CJK}

\end{document}
