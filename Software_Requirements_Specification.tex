\documentclass{scrreprt}

\usepackage{listings}
\usepackage{underscore}
\usepackage[bookmarks=true]{hyperref}
\usepackage[utf8]{inputenc}
\usepackage[english]{babel}
\usepackage{CJKutf8}
\usepackage{latexsym}
\usepackage{graphicx}
\usepackage[table]{xcolor}
\usepackage{blindtext}


\def\myversion{1.0 }
\date{}

\usepackage{hyperref}
\begin{document}

\begin{flushright}
    \rule{16cm}{5pt}\vskip1cm
    \begin{bfseries}
        \Huge{SOFTWARE REQUIREMENTS\\ SPECIFICATION}\\
        \vspace{1.9cm}
        for\\
        \vspace{1.9cm}
        $<$Final Project$>$\\
        \vspace{1.9cm}
        Prepared by $<$Team 2$>$\\
        \vspace{1.9cm}
        \today\\
    \end{bfseries}
\end{flushright}

\tableofcontents


\chapter{Introduction}

\section{Purpose}
%此model的用途
\begin{CJK}{UTF8}{bkai}
	此專題,主要用於辨識元智大學校園內的花卉,讓使用者在校園中看到喜歡的花朵,可以用此專題滿足好奇心、求知慾同時也能寓教於樂。但也能推廣到其他場所,只要提供足夠的 database便能進行training model \\[6pt]
	並且,有了這個模型,我們就不必記住花卉大量且不同的特徵,能夠輕易地識別出不同的品種同時又具有較高的辨別力。
\end{CJK}

\section{Intended Audience and Reading Suggestions}
%目標使用者及閱讀建議

\begin{CJK}{UTF8}{bkai}
	
\end{CJK}
\begin{enumerate}
\item
\begin{CJK}{UTF8}{bkai}
	使用者 : 主要面向為對於花卉鑑賞感興趣, 但無一定基礎的人,也可以用於教學。\\
如果數據量足夠大,或許能夠突變種是從哪種植物轉變而來,對植物相關領域有所幫助。
\end{CJK}
\item
\begin{CJK}{UTF8}{bkai}
	特色景點或區域的經營者 : 如果所經營的地方,有一定數量及種類的花朵,能過透過這個App提供給顧客一個新的樂趣進而提升景點知名度。
\end{CJK}

\end{enumerate}

\section{Project Scope}
\begin{CJK}{UTF8}{bkai}
 花朵辨識通常需要有一定的程度的相關知識及辨別力,為此需要大量的時間及精力學習,但也因為這樣可能會使人難以入門。
\end{CJK}
\begin{CJK}{UTF8}{bkai}
\\[6pt] 為了講求辨識速度及縮短training的過程,我們選擇以元智校園內出現的花朵為主,來訓練model。且訓練的data也是直接在學校內取材,以免因環境差異變化過大,導致辨識率下降。 \\[6pt]
這同時可以推廣到其他有特色的區域景點,鎖定一定範圍的區域來進行辨識,不僅能做到更精準的辨識與分類,且也能讓使用者知道在此區域能使用此App表示這邊有一定數量且不同種類花朵,無須漫無目的地去尋找,透過這個選擇,我們也能夠收到特定範圍內的使用者的回饋。
\end{CJK}

\chapter{Overall Description}

\section{Product Perspective}
%產品視角
\begin{CJK}{UTF8}{bkai}
整體而言,本專題的目標是作出一個元智大學校園內花朵的辨識系統。利用手機的攝像頭對準目標花朵,再對螢幕呈現出來的影像使用卷積神經網路(Convolutional Neural Network,CNN)進行分析,分析後將結果即該花資訊輸出在手機螢幕上。讓使用者能夠充分了解該花。
\end{CJK}

\section{Product Functions}
%產品功能.
\begin{CJK}{UTF8}{bkai}
	開啟App,將手機內的畫面對準要辨識的花朵,系統偵測到後,會立即在App上方顯示辨識出的品種及辨識度。
\end{CJK}

\section{User Classes and Characteristics}
%用戶群跟特徵
\begin{center}
	\begin{CJK}{UTF8}{bkai}
		
		\begin{tabular}{| l | l |}
		\hline
			\rowcolor{gray}User & Description \\ \hline
			學生及老師 & 可以做為課程教學使用  \\ \hline
			校內訪客 & 對於花朵鑑賞有所興趣 \\ \hline
		\end{tabular}
	\end{CJK}
\end{center}

\section{Operating Environment}
%操作環境
\begin{CJK}{UTF8}{bkai}
執行環境依不同對象分為兩種。 \\
\end{CJK}

\begin{CJK}{UTF8}{bkai}
使用者:
\end{CJK}
\begin{itemize}
\item
\begin{CJK}{UTF8}{bkai}
硬體需求:  有攝像鏡頭的智慧型手機或平板 
\end{CJK}
\item
\begin{CJK}{UTF8}{bkai}
軟體需求: Android  版本 8.0 
\end{CJK}
\end{itemize}

\begin{CJK}{UTF8}{bkai}
營運方: 
\end{CJK}
\begin{itemize}
\item
\begin{CJK}{UTF8}{bkai}
硬體需求:  桌上型或筆記電腦
\end{CJK}
\item
\begin{CJK}{UTF8}{bkai}
軟體需求:  Windows 10 
\end{CJK}
\item
\begin{CJK}{UTF8}{bkai}
執行環境: CPU
\end{CJK}
\end{itemize}

\section{Design and Implementation Constraints}
%設計和實作問題
\begin{CJK}{UTF8}{bkai}
	因本專題欲專注於辨識校園內的花朵,所以在資料收集方面只有拍攝校園內的花朵之外貌,因此,本模型在對同種花但是長相有很大的差異的花(花在不同環境下,顏色、樣貌等會略有不同),預測準度及信心度會較低。 \\[6pt]
	但是,從另一個角度來說,沒有其他環境下生長的花的資料,反而更能提升此模型對本校園中生長的花的預測準度。
\end{CJK}

\section{Assumptions and Dependencies}
%假設及依據
\begin{CJK}{UTF8}{bkai}
	本專題原本採用InceptionV3 的影像辨識模型,但是若要將他移植到APP上,model size太大了,且InceptionV3架構過於龐大,可能不利於行動裝置的即時辨識,因此我們改用MobileNet模型,其缺點就是準確度會稍比InceptionV3低,但是換來了更快的速度跟更小的空間。
\end{CJK}


\chapter{External Interface Requirements}

\section{User Interfaces}
%使用者介面
\begin{CJK}{UTF8}{bkai}
		已App為主體,把model建立起來與相機功能連結將目標儲存同時載入model分析。使用者使用App將鏡頭對準,畫面上面會及時地呈現出所掃描的花朵品種及辨識度。
\end{CJK}
\begin{figure}[h]
\begin{center}
\includegraphics[width=5cm]{userinterface2.jpg}
\end{center}
\caption{User Interface}
\end{figure}

\section{Hardware Interfaces}
%硬體介面
ASUS_Z012DA (Smart Phone)  \\
\begin{CJK}{UTF8}{bkai}
		以智慧型手機當作App 的載體。
\end{CJK}

\section{Software Interfaces}
%軟體介面
\subsection{TensorFlow package}
\begin{CJK}{UTF8}{bkai}
版本: 1.12.0 \\
		外掛式套件,將圖片進行image classification幫助我們做影像辨識。
\end{CJK}
\begin{figure}[h]
\begin{center}
\includegraphics[width=3cm]{tensorflow.jpg}
\end{center}
\caption{Tensorflow package}
\end{figure}

\subsection{Python}
\begin{CJK}{UTF8}{bkai}
版本: 2.7.12 \\
以python撰寫訓練model的程式,並輸出model。
\end{CJK}

\subsection{Android studio}
\begin{CJK}{UTF8}{bkai}
版本: 3.4 \\
		用anroid studio 來達成掃描花朵後存成資料在進行傳遞給model 來辨識,並將的回傳辨識度及品種資訊呈現出來。
\end{CJK}
\begin{figure}[h]
\begin{center}
\includegraphics[width=3cm]{androidstudioicon.jpg}
\end{center}
\caption{Android Studio}
\end{figure}


\chapter{System Features}

\section{Description and Priority}
%描述及優先權
%example: 用戶可以通過向網格添加課程對象來創建季度課程計劃,其中的列表示季度。 可以刪除課程並通過拖放在四分之一之間移動。 可以編輯課程數據。 用戶可以指定必備和必修課程以及課程實現的單元數。
\begin{CJK}{UTF8}{bkai}
		如果使用者移動鏡頭,需要偵測物件的轉換。而偵測由tensorflow來執行判斷。
\end{CJK}

\section{Stimulus/Response Sequences}
\begin{CJK}{UTF8}{bkai}
		Stimulus: 	將App內顯示的鏡頭對準物件,系統將自行掃描有無偵測到花朵 \\
		Response:  偵測後,進行辨識並顯示辨識結果
\end{CJK}

\section{Functional Requirements}
%功能需求
\begin{center}
	\begin{CJK}{UTF8}{bkai}
		
		\begin{tabular}{| l | l |}
		\hline
			\rowcolor{gray}Function name & Description \\ \hline
			Classifier creat &   建立一個 recognition的class \\ \hline
			recognizeImage & 將資料做轉換,傳入tensorflow進行運算,將最佳結果存入recognition中的變數 \\ \hline
			layout & tensorflow.demo.RecognitionScoreView \ : 在RecognitionScoreView.java中控制 \\ \hline
				 & 輸出內容, 從recognition中,叫出需要的變數內容,顯示到頁面上\\ \hline
		\end{tabular}
	\end{CJK}
\end{center}


\chapter{Other Nonfunctional Requirements}

\section{Performance Requirements}
%性能需求
\begin{CJK}{UTF8}{bkai}
	即時的(小於1s),準確的判斷出元智大學校園內花的種類( 趨近100\% )。
\end{CJK}

\end{document}
