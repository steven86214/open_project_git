\documentclass{scrreprt}

\usepackage{listings}
\usepackage{underscore}
\usepackage[bookmarks=true]{hyperref}
\usepackage[utf8]{inputenc}
\usepackage[english]{babel}

\def\myversion{1.0 }
\date{}

\usepackage{hyperref}
\begin{document}

\begin{flushright}
    \rule{16cm}{5pt}\vskip1cm
    \begin{bfseries}
        \Huge{SOFTWARE REQUIREMENTS\\ SPECIFICATION}\\
        \vspace{1.9cm}
        for\\
        \vspace{1.9cm}
        $<$Final Project$>$\\
        \vspace{1.9cm}
        Prepared by $<$Team 2$>$\\
        \vspace{1.9cm}
        \today\\
    \end{bfseries}
\end{flushright}

\tableofcontents


\chapter{Introduction}

\section{Purpose}
%此model的用途

\section{Intended Audience and Reading Suggestions}
%目標使用者及閱讀建議

\section{Project Scope}
%描述指定的軟件。並包括收益,目標和目標。這應該與整體業務目標相關,特別是如果開發之外的團隊可以訪問SRS。
%表徵產品,服務或結果的特徵和功能
%example: 該軟件面向需要創建季度課程安排的學生和行政人員,可供個人使用或分發給學生。 該軟件應允許用戶創建計劃,查看課程先決條件,並根據單獨的需求文件驗證計劃。
The software is targeted at both students and administrative faculty who need to create a quarterly course schedule, either for personal use or to be distributed to students. The software shall allow users to create a schedule, view course prerequisites, and validate a schedule against a separate requirements file.

\chapter{Overall Description}

\section{Product Perspective}
%產品視角
%example: 該軟件將是一個新的,獨立的產品。 它不是基於現有項目。
The software will be a new, self-contained product. It is not based on an existing project.

\section{Product Functions}
%產品功能

\section{User Classes and Characteristics}
%用戶群跟特徵

\section{Operating Environment}
%操作環境
%example
The system shall use the Java SE 6 platform. \\
The system shall use the Swing toolkit for GUI elements.

\section{Design and Implementation Constraints}
%設計和實作問題
%example: 用戶應指定特定課程滿足的要求。
Users shall specify which requirements specific courses fulfill.

\section{Assumptions and Dependencies}
%假設及依據
%example: 校園部門將提供一個XML文件,詳細說明了該部門特定目錄年後學生的要求。
Campus departments will provide an XML file detailing the requirements for students following a specific catalog year in that department. 

\chapter{External Interface Requirements}

\section{User Interfaces}
%使用者介面

\section{Hardware Interfaces}
%硬體介面

\section{Software Interfaces}
%軟體介面


\chapter{System Features}

\section{Description and Priority}
%描述及優先權
%example: 用戶可以通過向網格添加課程對象來創建季度課程計劃,其中的列表示季度。 可以刪除課程並通過拖放在四分之一之間移動。 可以編輯課程數據。 用戶可以指定必備和必修課程以及課程實現的單元數。
Users may create a quarterly course schedule by adding course objects to a grid, the columns of which represent quarters. Courses may be deleted and moved between quarters via drag-and-drop. Course data may be edited. Users may specify prerequisite and corequisite courses as well as the number of units the course fulfills.

\section{Stimulus/Response Sequences}
%一個request一個response
%example
Stimulus: User requests to create a new course in a specific quarter. \\
Response: System provides a form for the user to enter the course data. \\
Stimulus: User requests to delete a course. \\
Response: The course is removed from the flowchart. \\
Stimulus: User requests to update data for a course. \\
Response: The course data is presented in an editable format. When finished, \\
changes may be saved or discarded. \\
Stimulus: User requests to move a course to a different quarter. \\
Response: The course is moved to the column representing the desired quarter in the flowchar

\section{Functional Requirements}
%功能需求


\chapter{Other Nonfunctional Requirements}

\section{Performance Requirements}
%性能需求

\section{Safety Requirements}
%安全需求(optional)

\section{Security Requirements}
%安全需求(optional)

\end{document}
